\chapter{Introduction JEE}

\section{Kontrollfragen}

\subsection{Aus welchen Teilen besteht ein Java EE Server?}
Web Container, EJB Container, Application Client Container.

\subsection{Was ist ein Web Contrainer?}
\begin{itemize}
	\item Laufzeitumgebung für Servlets
	\item Web Container verwaltet das Life Cycle (Create, Destroy, Garbage Collector) der Beans
	\item Java Servlet / Java Server Faces
\end{itemize}

\subsection{Was ist ein EJB-Container?}
Schnittstelle zwischen Enterprise Bean (Business Logik) und Java EE Server. Verwaltet den Lifecycle von den EJBs und hat noch einige Aufgaben mehr.

\subsection{Was ist ein Application Client Server?}
Auch Fat Client genannt:
\begin{itemize}
	\item Der Application Client Server ist die Schnittstelle zwischen Java EE Application Clients und den Java EE Server.
	\item Java EE Clients sind spezielle Java SE Applikationen welche Java EE Server Komponenten verwenden
	\item Der Java Application Client läuft auf der Client Maschine und ist der Gateway zwischen der Client Applikation und den Java EE Komponenten die der Client verwenden will.
\end{itemize}

\subsection{Beschreiben Sie die Funktionalität der einzelnen Tiers in einer 3-Tier Architektur.}
\begin{itemize}
	\item Client Tier
	\item Middle Tier, aufgeteilt in Web und Business Tier
	\item Data Tier, EIS (Enterprise Information System Tier) nicht nur DB, kann auch ERP und andere Legacy Applikation sein.
\end{itemize}

\subsection{Nennen Sie ein paar Services die ihnen von einem JEE Container zur Verfügung gestellt werden. (MEP)}
\begin{itemize}
	\item Transaction Services
	\item Security Services 
	\item JNDA (Naming / Lookup Service)
	\item Remoting (RMI)
	\item Naming
	\item Instance Pooling
	\item Persistence
\end{itemize}

\subsection{Wieso werden die Klassen in Java EE nicht einfach Klassen genannt?}
Weil sie nicht vom Programmierer instanziiert werden. Sie werden vom Container verwaltet. \emph{Also generell sprechen wir immer noch von Klassen. Die Klassen welche vom Container verwaltet werden, haben oft spezielle Namen, wie EJBs, Servlet usw.}

\subsection{Was ist ein Domain Model?}
\begin{itemize}
	\item Funktionalitäten im Business
	\item Das Domain Model umfasst die wichtigsten Objekte im Kontext einer Business-Anwendung. Es repräsentiert die existierende Objekte oder Anwendung in einem Geschäftszweig.
\end{itemize}

\subsection{Was gehört zu einer guten Architektur?}
\begin{itemize}
	\item Konzeptuelle Integrität
	\item Korrektheit und Vollständigkeit
	\item Machbarkeit
\end{itemize}
